% Options for packages loaded elsewhere
\PassOptionsToPackage{unicode}{hyperref}
\PassOptionsToPackage{hyphens}{url}
%
\documentclass[
]{article}
\usepackage{amsmath,amssymb}
\usepackage{lmodern}
\usepackage{iftex}
\ifPDFTeX
  \usepackage[T1]{fontenc}
  \usepackage[utf8]{inputenc}
  \usepackage{textcomp} % provide euro and other symbols
\else % if luatex or xetex
  \usepackage{unicode-math}
  \defaultfontfeatures{Scale=MatchLowercase}
  \defaultfontfeatures[\rmfamily]{Ligatures=TeX,Scale=1}
\fi
% Use upquote if available, for straight quotes in verbatim environments
\IfFileExists{upquote.sty}{\usepackage{upquote}}{}
\IfFileExists{microtype.sty}{% use microtype if available
  \usepackage[]{microtype}
  \UseMicrotypeSet[protrusion]{basicmath} % disable protrusion for tt fonts
}{}
\makeatletter
\@ifundefined{KOMAClassName}{% if non-KOMA class
  \IfFileExists{parskip.sty}{%
    \usepackage{parskip}
  }{% else
    \setlength{\parindent}{0pt}
    \setlength{\parskip}{6pt plus 2pt minus 1pt}}
}{% if KOMA class
  \KOMAoptions{parskip=half}}
\makeatother
\usepackage{xcolor}
\IfFileExists{xurl.sty}{\usepackage{xurl}}{} % add URL line breaks if available
\IfFileExists{bookmark.sty}{\usepackage{bookmark}}{\usepackage{hyperref}}
\hypersetup{
  pdftitle={Pandoc Filter to Insert Arbitrary Complex Tables},
  hidelinks,
  pdfcreator={LaTeX via pandoc}}
\urlstyle{same} % disable monospaced font for URLs
\usepackage{color}
\usepackage{fancyvrb}
\newcommand{\VerbBar}{|}
\newcommand{\VERB}{\Verb[commandchars=\\\{\}]}
\DefineVerbatimEnvironment{Highlighting}{Verbatim}{commandchars=\\\{\}}
% Add ',fontsize=\small' for more characters per line
\newenvironment{Shaded}{}{}
\newcommand{\AlertTok}[1]{\textcolor[rgb]{1.00,0.00,0.00}{\textbf{#1}}}
\newcommand{\AnnotationTok}[1]{\textcolor[rgb]{0.38,0.63,0.69}{\textbf{\textit{#1}}}}
\newcommand{\AttributeTok}[1]{\textcolor[rgb]{0.49,0.56,0.16}{#1}}
\newcommand{\BaseNTok}[1]{\textcolor[rgb]{0.25,0.63,0.44}{#1}}
\newcommand{\BuiltInTok}[1]{#1}
\newcommand{\CharTok}[1]{\textcolor[rgb]{0.25,0.44,0.63}{#1}}
\newcommand{\CommentTok}[1]{\textcolor[rgb]{0.38,0.63,0.69}{\textit{#1}}}
\newcommand{\CommentVarTok}[1]{\textcolor[rgb]{0.38,0.63,0.69}{\textbf{\textit{#1}}}}
\newcommand{\ConstantTok}[1]{\textcolor[rgb]{0.53,0.00,0.00}{#1}}
\newcommand{\ControlFlowTok}[1]{\textcolor[rgb]{0.00,0.44,0.13}{\textbf{#1}}}
\newcommand{\DataTypeTok}[1]{\textcolor[rgb]{0.56,0.13,0.00}{#1}}
\newcommand{\DecValTok}[1]{\textcolor[rgb]{0.25,0.63,0.44}{#1}}
\newcommand{\DocumentationTok}[1]{\textcolor[rgb]{0.73,0.13,0.13}{\textit{#1}}}
\newcommand{\ErrorTok}[1]{\textcolor[rgb]{1.00,0.00,0.00}{\textbf{#1}}}
\newcommand{\ExtensionTok}[1]{#1}
\newcommand{\FloatTok}[1]{\textcolor[rgb]{0.25,0.63,0.44}{#1}}
\newcommand{\FunctionTok}[1]{\textcolor[rgb]{0.02,0.16,0.49}{#1}}
\newcommand{\ImportTok}[1]{#1}
\newcommand{\InformationTok}[1]{\textcolor[rgb]{0.38,0.63,0.69}{\textbf{\textit{#1}}}}
\newcommand{\KeywordTok}[1]{\textcolor[rgb]{0.00,0.44,0.13}{\textbf{#1}}}
\newcommand{\NormalTok}[1]{#1}
\newcommand{\OperatorTok}[1]{\textcolor[rgb]{0.40,0.40,0.40}{#1}}
\newcommand{\OtherTok}[1]{\textcolor[rgb]{0.00,0.44,0.13}{#1}}
\newcommand{\PreprocessorTok}[1]{\textcolor[rgb]{0.74,0.48,0.00}{#1}}
\newcommand{\RegionMarkerTok}[1]{#1}
\newcommand{\SpecialCharTok}[1]{\textcolor[rgb]{0.25,0.44,0.63}{#1}}
\newcommand{\SpecialStringTok}[1]{\textcolor[rgb]{0.73,0.40,0.53}{#1}}
\newcommand{\StringTok}[1]{\textcolor[rgb]{0.25,0.44,0.63}{#1}}
\newcommand{\VariableTok}[1]{\textcolor[rgb]{0.10,0.09,0.49}{#1}}
\newcommand{\VerbatimStringTok}[1]{\textcolor[rgb]{0.25,0.44,0.63}{#1}}
\newcommand{\WarningTok}[1]{\textcolor[rgb]{0.38,0.63,0.69}{\textbf{\textit{#1}}}}
\usepackage{longtable,booktabs,array}
\usepackage{calc} % for calculating minipage widths
% Correct order of tables after \paragraph or \subparagraph
\usepackage{etoolbox}
\makeatletter
\patchcmd\longtable{\par}{\if@noskipsec\mbox{}\fi\par}{}{}
\makeatother
% Allow footnotes in longtable head/foot
\IfFileExists{footnotehyper.sty}{\usepackage{footnotehyper}}{\usepackage{footnote}}
\makesavenoteenv{longtable}
% Make links footnotes instead of hotlinks:
\DeclareRobustCommand{\href}[2]{#2\footnote{\url{#1}}}
\setlength{\emergencystretch}{3em} % prevent overfull lines
\providecommand{\tightlist}{%
  \setlength{\itemsep}{0pt}\setlength{\parskip}{0pt}}
\setcounter{secnumdepth}{-\maxdimen} % remove section numbering
\usepackage{multirow}
\usepackage[normalem]{ulem}
\pdfstringdefDisableCommands{\renewcommand{\sout}{}}
\makeatletter
\@ifpackageloaded{subfig}{}{\usepackage{subfig}}
\@ifpackageloaded{caption}{}{\usepackage{caption}}
\captionsetup[subfloat]{margin=0.5em}
\AtBeginDocument{%
\renewcommand*\figurename{Figure}
\renewcommand*\tablename{Table}
}
\AtBeginDocument{%
\renewcommand*\listfigurename{List of Figures}
\renewcommand*\listtablename{List of Tables}
}
\newcounter{pandoccrossref@subfigures@footnote@counter}
\newenvironment{pandoccrossrefsubfigures}{%
\setcounter{pandoccrossref@subfigures@footnote@counter}{0}
\begin{figure}\centering%
\gdef\global@pandoccrossref@subfigures@footnotes{}%
\DeclareRobustCommand{\footnote}[1]{\footnotemark%
\stepcounter{pandoccrossref@subfigures@footnote@counter}%
\ifx\global@pandoccrossref@subfigures@footnotes\empty%
\gdef\global@pandoccrossref@subfigures@footnotes{{##1}}%
\else%
\g@addto@macro\global@pandoccrossref@subfigures@footnotes{, {##1}}%
\fi}}%
{\end{figure}%
\addtocounter{footnote}{-\value{pandoccrossref@subfigures@footnote@counter}}
\@for\f:=\global@pandoccrossref@subfigures@footnotes\do{\stepcounter{footnote}\footnotetext{\f}}%
\gdef\global@pandoccrossref@subfigures@footnotes{}}
\@ifpackageloaded{float}{}{\usepackage{float}}
\floatstyle{ruled}
\@ifundefined{c@chapter}{\newfloat{codelisting}{h}{lop}}{\newfloat{codelisting}{h}{lop}[chapter]}
\floatname{codelisting}{Listing}
\newcommand*\listoflistings{\listof{codelisting}{List of Listings}}
\makeatother
\ifLuaTeX
  \usepackage{selnolig}  % disable illegal ligatures
\fi

\title{Pandoc Filter to Insert Arbitrary Complex Tables}
\author{}
\date{}

\begin{document}
\maketitle

Outputs: \href{https://yongfu.name/insertTables/}{Web Page} /
\href{https://yongfu.name/insertTables/README.tex}{LaTeX} /
\href{https://yongfu.name/insertTables/README.pdf}{PDF} /
\href{https://www.overleaf.com/docs?snip_uri=https://yongfu.name/insertTables/README.tex\&engine=xelatex}{Overleaf}

\hypertarget{dependencies}{%
\subsection{Dependencies}\label{dependencies}}

Make sure you have Pandoc and
\href{https://github.com/lierdakil/pandoc-crossref}{pandoc-crossref}
installed (callable from cmd).

\hypertarget{usage}{%
\subsection{Usage}\label{usage}}

Write your complex tables in HTML in \texttt{tables.html} and in LaTeX
in \texttt{tables.tex}. \url{https://tablesgenerator.com} is a good
resource for constructing complex tables. To insert tables into the
output HTML/LaTeX document, use the syntax
\texttt{\textless{}COMMENT\textgreater{}\ tbl:table-id\ \textless{}COMMENT\textgreater{}}
to mark the beginning and
\texttt{\textless{}COMMENT\textgreater{}\ END\ \textless{}COMMENT\textgreater{}}
to mark the end of a table definition in \texttt{tables.html} and
\texttt{tables.tex}. \texttt{\textless{}COMMENT\textgreater{}}
corresponds to \texttt{\%} in LaTeX and \texttt{\textless{}!-\/-} and
\texttt{-\/-\textgreater{}} in HTML. \texttt{tbl:table-id} is the
identifier of the table used for cross-referencing in the markdown
source. Refer to
\href{https://github.com/lierdakil/pandoc-crossref}{pandoc-crossref} for
details of cross referencing tables.

To compile the documents, apply the filter \texttt{custom-table.py}
\textbf{AFTER} \texttt{pandoc-crossref} in the command line.

\begin{Shaded}
\begin{Highlighting}[]
\ExtensionTok{pandoc} \AttributeTok{{-}F}\NormalTok{ pandoc{-}crossref }\AttributeTok{{-}{-}lua{-}filter}\NormalTok{ insertTables.lua README.md }\AttributeTok{{-}o}\NormalTok{ README.tex}
\ExtensionTok{pandoc} \AttributeTok{{-}F}\NormalTok{ pandoc{-}crossref }\AttributeTok{{-}{-}lua{-}filter}\NormalTok{ insertTables.lua README.md }\AttributeTok{{-}o}\NormalTok{ README.html}
\end{Highlighting}
\end{Shaded}

\hypertarget{example}{%
\subsection{Example}\label{example}}

%%%%%%% tbl:custom-table %%%%%%%%
\begin{table}[!htb]
    \centering
    \caption{\label{tbl:custom-table}This is a \emph{complex table}, \textbf{written} in \texttt{tables.tex} and \texttt{tables.html}.}
\begin{tabular}{lllll}
        \hline
        \textbf{} & \multicolumn{4}{l}{Column Span} \\ \hline
        \multirow{2}{*}{Row Span} & a & b & d & f \\
         & c & d & e & g \\ \hline
        \end{tabular}
\end{table}
%%%%%%% END %%%%%%%%

See Table~\ref{tbl:custom-table}.

\hypertarget{tbl:normal-table}{}
\begin{longtable}[]{@{}ll@{}}
\caption{\label{tbl:normal-table}This is a normal table written in
markdown, which will not be replaced.}\tabularnewline
\toprule
Column A & Column B \\
\midrule
\endfirsthead
\toprule
Column A & Column B \\
\midrule
\endhead
A1 & B1 \\
A2 & B2 \\
\bottomrule
\end{longtable}

\hypertarget{custom-caption-positions}{%
\subsubsection{Custom Caption
Positions}\label{custom-caption-positions}}

By default, \texttt{insertTables.lua} looks for the string
\texttt{\textbackslash{}begin\{tabular} and inserts the caption before
it. In circumstances where \texttt{\textbackslash{}begin\{tabular\}} or
\texttt{\textbackslash{}begin\{tabularx\}} are not present in the
table's code, this filter will fail. To deal with these cases, you have
to tell \texttt{insertTables.lua} where to insert the caption by placing
the anchor \texttt{\%caption\%} in your table's code. This may also be
useful when you want to place the caption \textbf{below} the table body.
This can be achieved by placing the anchor \texttt{\%caption\%}
\textbf{after} the \texttt{tabular} environment:

\begin{Shaded}
\begin{Highlighting}[]
\KeywordTok{\textbackslash{}begin}\NormalTok{\{}\ExtensionTok{table}\NormalTok{\}[!htb]}
    \FunctionTok{\textbackslash{}centering}
    \KeywordTok{\textbackslash{}begin}\NormalTok{\{}\ExtensionTok{tabular}\NormalTok{\}\{lllll\}}
        \FunctionTok{\textbackslash{}hline}
        \FunctionTok{\textbackslash{}textbf}\NormalTok{\{\} }\OperatorTok{\&} \FunctionTok{\textbackslash{}multicolumn}\NormalTok{\{4\}\{l\}\{Column Span\} }\FunctionTok{\textbackslash{}\textbackslash{}} \FunctionTok{\textbackslash{}hline}
        \FunctionTok{\textbackslash{}multirow}\NormalTok{\{2\}\{*\}\{Row Span\} }\OperatorTok{\&}\NormalTok{ a }\OperatorTok{\&}\NormalTok{ b }\OperatorTok{\&}\NormalTok{ d }\OperatorTok{\&}\NormalTok{ f }\FunctionTok{\textbackslash{}\textbackslash{}}
         \OperatorTok{\&}\NormalTok{ c }\OperatorTok{\&}\NormalTok{ d }\OperatorTok{\&}\NormalTok{ e }\OperatorTok{\&}\NormalTok{ g }\FunctionTok{\textbackslash{}\textbackslash{}} \FunctionTok{\textbackslash{}hline}
        \KeywordTok{\textbackslash{}end}\NormalTok{\{}\ExtensionTok{tabular}\NormalTok{\}}
    \CommentTok{\%caption\%}
\KeywordTok{\textbackslash{}end}\NormalTok{\{}\ExtensionTok{table}\NormalTok{\}}
\end{Highlighting}
\end{Shaded}

which results in Table~\ref{tbl:custom-caption-position} (this could
only be seen in \href{https://yongfu.name/insertTables/README.tex}{tex}
or \href{https://yongfu.name/insertTables/README.pdf}{PDF} output).

%%%%%%% tbl:custom-caption-position %%%%%%%%
\begin{table}[!htb]
    \centering
    \begin{tabular}{lllll}
        \hline
        \textbf{} & \multicolumn{4}{l}{Column Span} \\ \hline
        \multirow{2}{*}{Row Span} & a & b & d & f \\
         & c & d & e & g \\ \hline
        \end{tabular}
    \caption{\label{tbl:custom-caption-position}For LaTeX tables, you can define the position of the caption with the string \texttt{\%caption\%} in \texttt{tables.tex}.}
\end{table}
%%%%%%% END %%%%%%%%

\end{document}
